\documentclass[a4paper,12pt]{article}
\usepackage{graphicx}
\usepackage{enumitem}
\usepackage{titlesec}
\usepackage{geometry}

% Title format
\titleformat{\section}{\large\bfseries}{}{0em}{}

% Page margins
\geometry{left=3cm, right=3cm, top=3cm, bottom=3cm}

\title{Master Thesis Proposal Template with Time Management Strategy}

\author{Lukas Monrad-Krohn}

\date{\today}

\begin{document}

\maketitle

\section*{Instructions}
This template provides a structured outline for your master thesis proposal and includes a time management exercise. Complete each section with brief responses (1–2 sentences or bullet points) to save time. After completing, push your updates to GitHub as practice.

\section{1. Summary}
What is the main purpose of your research, and why is it significant?\\
Sublimation of blowing snow has a large impact on surface mass balance of snow, latent heat fluxes in the ABL and INP production rates in the polar areas. The Dery and Yau (2001) parameterization is often used to quantify the sublimation rate, but it is unknown how accurate it is. That is why I want to compare the parameterized sublimation rate with observations from the MOSAiC campaign and NYA to evaluate it.

\section{2. Research Background}
What previous research provides context for your study, and what gaps does it address?\\
No research has been done to evaluate this parameterization in the central Arctic and NYA. It has been evaluated with model results from the Canadian prairie, but is used in the Arctic Ocean and Antarctica.

\section{3. Research Objectives (incl. Research Questions)}
State the main objectives of your research and outline specific research questions.
How large is the sublimation rate from blowing snow in NYA and during MOSAiC? How does teh observed sublimation rate compare with the parameterized ones?

\section{4. Methods and Data}
Which methods will you use to answer your research questions, and why are they suitable?\\
Observations of blowing snow with a snow particle counter and of the latent heat flux with sonic anemometers and high freq. water vapour content measurements. Combining the observations with output form a 1D- non-steady blowing snow model to get a comparable column integrated sublimation rate. 

\section{5. Timeline and Milestones (incl. Gantt Chart)}
What are the main stages of your research, and when do you plan to complete them?\\
1. Literature review
2. Fieldwork
3. Data analysis
4. Writing

\section{Exercise: Time Management Strategy}
Use this section to outline your weekly planning, prioritisation, and reflection. Keep responses short and focused.

\subsection*{Weekly Planning}
\textbf{Prompt:} List 1–2 key tasks you need to complete this week.\\
\textbf{Answer:}
\begin{itemize}
    \item book flights
    \item sort literature
    \item plan installation of SPC
    \item check calibration data
    \item email Markus
    \item email Karsten, Stephen, Norbert
\end{itemize}

\subsection*{Task Prioritisation}
 Categorise a few tasks as: 1. Urgent and Important, 2. Important but Not Urgent, 3. Urgent but Not Important, 4. Not Urgent and Not Important.\\
\textbf{Answer:}
\begin{itemize}
    \item Urgent and Important: email Markus, installation of SPC
    \item Important but Not Urgent: literature
    \item Urgent but Not Important: flights
    \item Not Urgent and Not Important: email others, calibration
\end{itemize}

\subsection*{Reflection and Adjustment}
Briefly reflect on your progress this week. What went well? What challenges did you face?\\
\textbf{Answer:} 
Logistics and planning are taking an aweful lot of work, but the stay is booked and solutions for shipping and installation found.
\noindent
\textbf{Adjustment for Next Week:} 
Get more actual work done (sorting the literature I already read).

\end{document}
